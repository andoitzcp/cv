\documentclass[10pt, letterpaper]{article}

% Packages:
\usepackage[
    ignoreheadfoot, % set margins without considering header and footer
    top=2 cm, % seperation between body and page edge from the top
    bottom=2 cm, % seperation between body and page edge from the bottom
    left=2 cm, % seperation between body and page edge from the left
    right=2 cm, % seperation between body and page edge from the right
    footskip=1.0 cm, % seperation between body and footer
    % showframe % for debugging
]{geometry} % for adjusting page geometry
\usepackage{titlesec} % for customizing section titles
\usepackage{tabularx} % for making tables with fixed width columns
\usepackage{array} % tabularx requires this
\usepackage[dvipsnames]{xcolor} % for coloring text
\definecolor{primaryColor}{RGB}{0, 79, 144} % define primary color
\usepackage{enumitem} % for customizing lists
\usepackage{fontawesome5} % for using icons
\usepackage{amsmath} % for math
\usepackage[
    pdftitle={CV Andoitz Campo},
    pdfauthor={Andoitz Campo},
    colorlinks=true,
    urlcolor=primaryColor
]{hyperref} % for links, metadata and bookmarks
\usepackage[pscoord]{eso-pic} % for floating text on the page
\usepackage{calc} % for calculating lengths
\usepackage{bookmark} % for bookmarks
\usepackage{lastpage} % for getting the total number of pages
\usepackage{changepage} % for one column entries (adjustwidth environment)
\usepackage{paracol} % for two and three column entries
\usepackage{ifthen} % for conditional statements
\usepackage{needspace} % for avoiding page brake right after the section title
\usepackage{iftex} % check if engine is pdflatex, xetex or luatex

% Ensure that generate pdf is machine readable/ATS parsable:
\ifPDFTeX
    \input{glyphtounicode}
    \pdfgentounicode=1
    % \usepackage[T1]{fontenc} % this breaks sb2nov
    \usepackage[utf8]{inputenc}
    \usepackage{lmodern}
\fi



% Some settings:
\AtBeginEnvironment{adjustwidth}{\partopsep0pt} % remove space before adjustwidth environment
\pagestyle{empty} % no header or footer
\setcounter{secnumdepth}{0} % no section numbering
\setlength{\parindent}{0pt} % no indentation
\setlength{\topskip}{0pt} % no top skip
\setlength{\columnsep}{0cm} % set column seperation
\makeatletter
\let\ps@customFooterStyle\ps@plain % Copy the plain style to customFooterStyle
\patchcmd{\ps@customFooterStyle}{\thepage}{
    \color{gray}\textit{\small John Doe - Page \thepage{} of \pageref*{LastPage}}
}{}{} % replace number by desired string
\makeatother
\pagestyle{customFooterStyle}

\titleformat{\section}{\needspace{4\baselineskip}\bfseries\large}{}{0pt}{}[\vspace{1pt}\titlerule]

\titlespacing{\section}{
    % left space:
    -1pt
}{
    % top space:
    0.3 cm
}{
    % bottom space:
    0.2 cm
} % section title spacing

\renewcommand\labelitemi{$\circ$} % custom bullet points
\newenvironment{highlights}{
    \begin{itemize}[
        topsep=0.10 cm,
        parsep=0.10 cm,
        partopsep=0pt,
        itemsep=0pt,
        leftmargin=0.4 cm + 10pt
    ]
}{
    \end{itemize}
} % new environment for highlights

\newenvironment{highlightsforbulletentries}{
    \begin{itemize}[
        topsep=0.10 cm,
        parsep=0.10 cm,
        partopsep=0pt,
        itemsep=0pt,
        leftmargin=10pt
    ]
}{
    \end{itemize}
} % new environment for highlights for bullet entries


\newenvironment{onecolentry}{
    \begin{adjustwidth}{
        0.2 cm + 0.00001 cm
    }{
        0.2 cm + 0.00001 cm
    }
}{
    \end{adjustwidth}
} % new environment for one column entries

\newenvironment{twocolentry}[2][]{
    \onecolentry
    \def\secondColumn{#2}
    \setcolumnwidth{\fill, 4.5 cm}
    \begin{paracol}{2}
}{
    \switchcolumn \raggedleft \secondColumn
    \end{paracol}
    \endonecolentry
} % new environment for two column entries

\newenvironment{header}{
    \setlength{\topsep}{0pt}\par\kern\topsep\centering\linespread{1.5}
}{
    \par\kern\topsep
} % new environment for the header

\newcommand{\placelastupdatedtext}{% \placetextbox{<horizontal pos>}{<vertical pos>}{<stuff>}
  \AddToShipoutPictureFG*{% Add <stuff> to current page foreground
    \put(
        \LenToUnit{\paperwidth-2 cm-0.2 cm+0.05cm},
        \LenToUnit{\paperheight-1.0 cm}
    ){\vtop{{\null}\makebox[0pt][c]{
        \small\color{gray}\textit{Last updated in September 2024}\hspace{\widthof{Last updated in September 2024}}
    }}}%
  }%
}%

% save the original href command in a new command:
\let\hrefWithoutArrow\href

% new command for external links:
\renewcommand{\href}[2]{\hrefWithoutArrow{#1}{\ifthenelse{\equal{#2}{}}{ }{#2 }\raisebox{.15ex}{\footnotesize \faExternalLink*}}}


\begin{document}
\newcommand{\AND}{\unskip
\cleaders\copy\ANDbox\hskip\wd\ANDbox
\ignorespaces
}
\newsavebox\ANDbox
\sbox\ANDbox{}

\placelastupdatedtext
\begin{header}
  \textbf{\fontsize{24 pt}{24 pt}\selectfont Andoitz Campo}

  \vspace{0.3 cm}

  \normalsize
  \mbox{{\color{black}\footnotesize\faIdCard*}\hspace*{0.13cm}79171246W}%
  \kern 0.25 cm%
  \AND%
  \kern 0.25 cm%
  \mbox{\hrefWithoutArrow{mailto:acp@adotcpio.org}{\color{black}{\footnotesize\faEnvelope[regular]}\hspace*{0.13cm}acp@adotcpio.org}}%
  \kern 0.25 cm%
  \AND%
  \kern 0.25 cm%
  \mbox{\hrefWithoutArrow{tel: 648 80 42 83}{\color{black}{\footnotesize\faPhone*}\hspace*{0.13cm}648 80 42 83}}%
  \kern 0.25 cm%
  \AND%
  \kern 0.25 cm%
        \mbox{\hrefWithoutArrow{https://www.linkedin.com/in/andoitz-campo-1b2b631a9/}{\color{black}{\footnotesize\faLinkedinIn}\hspace*{0.13cm}andoitz-campo}}%
   \kern 0.25 cm%
   \AND%
   \kern 0.25 cm%
   \mbox{\hrefWithoutArrow{https://github.com/andoitzcp}{\color{black}{\footnotesize\faGithub}\hspace*{0.13cm}andoitzcp}}%
\end{header}

\vspace{0.3 cm - 0.3 cm}


\section{INTRODUCCIÓN}
    \begin{onecolentry}
          Aficionado a la tecnología que programa en su tiempo libre. Me dedico
          al tratamiento del dato, principalmente a la extracion de informacion
          de los modulos financieros y de RRHH de SAP en clientes como Iberdrola
          y Laboral Kutxa.
          Actualmente estoy liderando la implantacion de una ETL de SAP S4 a R3.
          En mi dia a dia me apoyo en lenguajes de scripting como python y
          powershell para sacar adelante grandes volumenes de trabajo.
    \end{onecolentry}

\section{HABILIDADES TÉCNICAS}

    \begin{onecolentry}
        \textbf{Lenguajes:} C, Python, SQL,
    \end{onecolentry}

    \vspace{0.2 cm}
    \begin{onecolentry}
        \textbf{Tecnologias:} SAP, SAC, Fiori, Microsoft SQL Server, Linux
    \end{onecolentry}

    \vspace{0.2 cm}
    \begin{onecolentry}
        \textbf{Librerías:} Pandas, Numpy, Tkinter, Selenium, Simpy, Manim
    \end{onecolentry}

    \vspace{0.2 cm}
    \begin{onecolentry}
        \textbf{Herramientas de desarrollador:} Git, AWK/SED, GNU Core tools, Vim, Emacs
    \end{onecolentry}

\section{EXPERIENCIA}
    \begin{twocolentry}{
        \textit{Feb 2023 – Act}}
            \textbf{Analista Técnico}
            \textit{Stratesys}
    \end{twocolentry}

    \vspace{0.10 cm}
    \begin{onecolentry}
        \begin{highlights}
            \item Desarrollo de una ETL en SQL Server para realizar una carga incremental de informacion de RRHH y ticketBAI proviniente de SAP S4 On Cloud.
            \item Desarrollo y mantenimiento de queries CDS a BDD SAP HANA consumidas en SAP Analytics Cloud (SAC).
            \item Automatización de la verificación y validación del dato mediante Python.
            \item Desarrollo de herramientas de web scraping usando la librería Selenium de Python.
            \item Desarrollo de herramientas de líneas de comando para automatización de tareas recurrentes.
            \item Tratamiento y extracción de dato en formato texto usando AWK/SED y GNU Core Tools.
            \item Reporting de información financiera en SAC.
        \end{highlights}
    \end{onecolentry}

    \vspace{0.2 cm}
    \begin{twocolentry}{
        \textit{May 2021 – Ago 2022}}
            \textbf{Técnico de Laboratorio}
            \textit{Bridgestone}
    \end{twocolentry}

    \vspace{0.10 cm}
    \begin{onecolentry}
        \begin{highlights}
            \item Gestión del laboratorio de calidad, análisis de la producción y generación de reportes.
            \item Análisis estadístico de resultados históricos de ensayos.
            \item Cálculo de KPIs y estimaciones de producción.
            \item Automatización del proceso de planificación de ensayos.
        \end{highlights}
    \end{onecolentry}

    \vspace{0.2 cm}
    \begin{twocolentry}{
        \textit{Oct 2020 – Dic 2020}}
            \textbf{Becario de Gestión de Calidad}
            \textit{Fluytec}
    \end{twocolentry}

    \vspace{0.10 cm}
    \begin{onecolentry}
        \begin{highlights}
             \item Generación de informes y gestión de No Conformidades.
        \end{highlights}
    \end{onecolentry}


\section{EDUCACIÓN}
    \begin{twocolentry}{
        \textit{2015– 2023}}
            \textbf{UPV/EHU}
            \textit{Grado en Ingeniería Química, Facultad de Ciencia y Tecnología}
    \end{twocolentry}

    \vspace{0.10 cm}
    \begin{twocolentry}{
        \textit{2022 – 2023}}
            \textbf{42 Foundation}
            \textit{Bootcamp de programación}
    \end{twocolentry}

\section{PUBLICACIONES}
    \begin{samepage}
        \begin{twocolentry}{May 2024}
            \textbf{Simulación de Eventos Discretos (DES) como método para la optimización de recursos en un laboratorio de evaluación de neumáticos.}
            \vspace{0.10 cm}

            \mbox{Andoitz Campo}
        \end{twocolentry}

        \vspace{0.20 cm}

        \begin{onecolentry}
        \href{http://hdl.handle.net/10810/67870}{http://hdl.handle.net/10810/67870}
            \end{onecolentry}
    \end{samepage}

\section{PROYECTOS}
    \begin{onecolentry}
        \textbf{SAC Report Scraper}
    \end{onecolentry}

    \vspace{0.10 cm}
    \begin{onecolentry}
        \begin{highlights}
            \item Desarrollé un bot que navegaba más de 30 informes y generaba alrededor de 5000 ejecuciones para generar los documentos de auditorías fiscales para la empresa.
            \item El ahorro fue equivalente a un equipo de 4 personas trabajando durante 3 semanas.
            \item Tecnologias: Python, Selenium, Git
        \end{highlights}
    \end{onecolentry}

    \vspace{0.2 cm}
    \begin{onecolentry}
        \textbf{ETL SAP S4 On Cloud - SAP R3}
    \end{onecolentry}

    \vspace{0.10 cm}
    \begin{onecolentry}
        \begin{highlights}
            \item El proyecto plantea una solucion a un escenario donde el
              cliente posee una BDD que se conecta directamente a la BDD de SAP
              R3 que sera sustituido por un SAP S4 en la nube. La solucion
              permite que la logica de negocio contenida en la BDD custom del
              cliente se mantenga mediante una carga incremental desde el nuevo S4 al
              R3.
            \item Disene e implemente la arquitectura de la carga incremental de
              mas de 80 tablas de SAP.
            \item Lidere un equipo de 2 tecnicos en el desarrollo del programa
              ABAP de exportacion de datos en formato XML.
            \item Tecnologias: SQL Server, PowerShell, python, ABAP.
        \end{highlights}
    \end{onecolentry}


\end{document}
