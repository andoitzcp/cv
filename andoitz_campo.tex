\documentclass[10pt, letterpaper]{article}
% Packages:
\usepackage[
    ignoreheadfoot,
    top=2 cm,
    bottom=2 cm,
    left=2 cm,
    right=2 cm,
    footskip=1.0 cm,
]{geometry}
\usepackage{titlesec}
\usepackage{tabularx}
\usepackage{array}
\usepackage[dvipsnames]{xcolor}
\definecolor{primaryColor}{RGB}{0, 79, 144}
\usepackage{enumitem}
\usepackage{fontawesome5}
\usepackage{amsmath}
\usepackage[
    pdftitle={CV Andoitz Campo},
    pdfauthor={Andoitz Campo},
    colorlinks=true,
    urlcolor=primaryColor
]{hyperref}
\usepackage[pscoord]{eso-pic}
\usepackage{calc}
\usepackage{bookmark}
\usepackage{lastpage}
\usepackage{changepage}
\usepackage{paracol}
\usepackage{ifthen}
\usepackage{needspace}
\usepackage{iftex}

\ifPDFTeX
    \input{glyphtounicode}
    \pdfgentounicode=1
    \usepackage[utf8]{inputenc}
    \usepackage{lmodern}
\fi

\AtBeginEnvironment{adjustwidth}{\partopsep0pt}
\pagestyle{empty}
\setcounter{secnumdepth}{0}
\setlength{\parindent}{0pt}
\setlength{\topskip}{0pt}
\setlength{\columnsep}{0cm}

\makeatletter
\let\ps@customFooterStyle\ps@plain
\patchcmd{\ps@customFooterStyle}{\thepage}{
    \color{gray}\textit{\small Andoitz Campo - Página \thepage{} de \pageref*{LastPage}}
}{}{}
\makeatother
\pagestyle{customFooterStyle}

\titleformat{\section}{\needspace{4\baselineskip}\bfseries\large}{}{0pt}{}[\vspace{1pt}\titlerule]
\titlespacing{\section}{
    -1pt
}{
    0.3 cm
}{
    0.2 cm
}

\renewcommand\labelitemi{$\circ$}

\newenvironment{highlights}{
    \begin{itemize}[
        topsep=0.10 cm,
        parsep=0.10 cm,
        partopsep=0pt,
        itemsep=0pt,
        leftmargin=0.4 cm + 10pt
    ]
}{
    \end{itemize}
}

\newenvironment{highlightsforbulletentries}{
    \begin{itemize}[
        topsep=0.10 cm,
        parsep=0.10 cm,
        partopsep=0pt,
        itemsep=0pt,
        leftmargin=10pt
    ]
}{
    \end{itemize}
}

\newenvironment{onecolentry}{
    \begin{adjustwidth}{
        0.2 cm + 0.00001 cm
    }{
        0.2 cm + 0.00001 cm
    }
}{
    \end{adjustwidth}
}

\newenvironment{twocolentry}[2][]{
    \onecolentry
    \def\secondColumn{#2}
    \setcolumnwidth{\fill, 4.5 cm}
    \begin{paracol}{2}
}{
    \switchcolumn \raggedleft \secondColumn
    \end{paracol}
    \endonecolentry
}

\newenvironment{header}{
    \setlength{\topsep}{0pt}\par\kern\topsep\centering\linespread{1.5}
}{
    \par\kern\topsep
}

\newcommand{\placelastupdatedtext}{
  \AddToShipoutPictureFG*{
    \put(
        \LenToUnit{\paperwidth-2 cm-0.2 cm+0.05cm},
        \LenToUnit{\paperheight-1.0 cm}
    ){\vtop{{\null}\makebox[0pt][c]{
        \small\color{gray}\textit{Última actualización en noviembre de 2025}\hspace{\widthof{Última actualización en noviembre de 2025}}
    }}}%
  }%
}

\let\hrefWithoutArrow\href
\renewcommand{\href}[2]{\hrefWithoutArrow{#1}{\ifthenelse{\equal{#2}{}}{ }{#2 }\raisebox{.15ex}{\footnotesize \faExternalLink*}}}

\begin{document}
\newcommand{\AND}{\unskip
\cleaders\copy\ANDbox\hskip\wd\ANDbox
\ignorespaces
}
\newsavebox\ANDbox
\sbox\ANDbox{}

\placelastupdatedtext

\begin{header}
  \textbf{\fontsize{24 pt}{24 pt}\selectfont Andoitz Campo}

  \vspace{0.3 cm}
  \normalsize
  \mbox{{\color{black}\footnotesize\faIdCard*}\hspace*{0.13cm}79171246W}%
  \kern 0.25 cm%
  \AND%
  \kern 0.25 cm%
  \mbox{\hrefWithoutArrow{mailto:acp@adotcpio.com}{\color{black}{\footnotesize\faEnvelope[regular]}\hspace*{0.13cm}acp@adotcpio.com}}%
  \kern 0.25 cm%
  \AND%
  \kern 0.25 cm%
  \mbox{\hrefWithoutArrow{tel:648804283}{\color{black}{\footnotesize\faPhone*}\hspace*{0.13cm}648 80 42 83}}%
  \kern 0.25 cm%
  \AND%
  \kern 0.25 cm%
  \mbox{\hrefWithoutArrow{https://www.linkedin.com/in/andoitz-campo/}{\color{black}{\footnotesize\faLinkedinIn}\hspace*{0.13cm}andoitz-campo}}%
  \kern 0.25 cm%
  \AND%
  \kern 0.25 cm%
  \mbox{\hrefWithoutArrow{https://github.com/andoitzcp}{\color{black}{\footnotesize\faGithub}\hspace*{0.13cm}andoitzcp}}%
\end{header}

\vspace{0.3 cm - 0.3 cm}

\section{INTRODUCCIÓN}
\begin{onecolentry}
    Aficionado a la tecnología que programa en su tiempo libre. Me dedico al tratamiento de datos, principalmente a la extracción de información de los módulos financieros y de Recursos Humanos de SAP en clientes como Iberdrola y Laboral Kutxa. Actualmente estoy liderando la implementación de una ETL de SAP S/4 a R3. En mi día a día, me apoyo en lenguajes de scripting como Python y PowerShell para manejar grandes volúmenes de trabajo.
\end{onecolentry}

\section{HABILIDADES TÉCNICAS}
\begin{onecolentry}
    \textbf{Lenguajes:} C, Python, SQL
\end{onecolentry}
\vspace{0.2 cm}
\begin{onecolentry}
    \textbf{Tecnologías:} SAP, SAC, Fiori, Microsoft SQL Server, Linux
\end{onecolentry}
\vspace{0.2 cm}
\begin{onecolentry}
    \textbf{Librerías:} Pandas, NumPy, Tkinter, Selenium, Simpy, Manim
\end{onecolentry}
\vspace{0.2 cm}
\begin{onecolentry}
    \textbf{Herramientas de desarrollador:} Git, AWK/SED, GNU Core Tools, Vim, Emacs
\end{onecolentry}

\section{EXPERIENCIA}
\begin{twocolentry}{\textit{Feb 2023 – Actualidad}}
    \textbf{Analista Técnico} \textit{Stratesys}
\end{twocolentry}
\vspace{0.10 cm}
\begin{onecolentry}
    \begin{highlights}
        \item Desarrollo de una ETL en SQL Server para realizar una carga incremental de información de Recursos Humanos y TicketBAI proveniente de SAP S/4HANA Cloud.
        \item Desarrollo y mantenimiento de consultas CDS en bases de datos SAP HANA consumidas en SAP Analytics Cloud (SAC).
        \item Automatización de la verificación y validación de datos mediante Python.
        \item Desarrollo de herramientas de web scraping usando la librería Selenium de Python.
        \item Desarrollo de herramientas de línea de comandos para la automatización de tareas recurrentes.
        \item Tratamiento y extracción de datos en formato de texto usando AWK/SED y GNU Core Tools.
        \item Generación de informes financieros en SAC.
    \end{highlights}
\end{onecolentry}

\vspace{0.2 cm}
\begin{twocolentry}{\textit{May 2021 – Ago 2022}}
    \textbf{Técnico de Laboratorio} \textit{Bridgestone}
\end{twocolentry}
\vspace{0.10 cm}
\begin{onecolentry}
    \begin{highlights}
        \item Gestión del laboratorio de calidad, análisis de la producción y generación de informes.
        \item Análisis estadístico de resultados históricos de ensayos.
        \item Cálculo de KPIs y estimaciones de producción.
        \item Automatización del proceso de planificación de ensayos.
    \end{highlights}
\end{onecolentry}

\vspace{0.2 cm}
\begin{twocolentry}{\textit{Oct 2020 – Dic 2020}}
    \textbf{Becario de Gestión de Calidad} \textit{Fluytec}
\end{twocolentry}
\vspace{0.10 cm}
\begin{onecolentry}
    \begin{highlights}
        \item Generación de informes y gestión de No Conformidades.
    \end{highlights}
\end{onecolentry}

\section{EDUCACIÓN}
\begin{twocolentry}{\textit{2015 – 2023}}
    \textbf{UPV/EHU} \textit{Grado en Ingeniería Química, Facultad de Ciencia y Tecnología}
\end{twocolentry}
\vspace{0.10 cm}
\begin{twocolentry}{\textit{2022 – 2023}}
    \textbf{42 Foundation} \textit{Bootcamp de programación}
\end{twocolentry}

\section{PUBLICACIONES}
\begin{samepage}
    \begin{twocolentry}{May 2024}
        \textbf{Simulación de Eventos Discretos (DES) como método para la optimización de recursos en un laboratorio de evaluación de neumáticos.}
        \vspace{0.10 cm}

        \mbox{Andoitz Campo}
    \end{twocolentry}
    \vspace{0.20 cm}
    \begin{onecolentry}
        \href{http://hdl.handle.net/10810/67870}{http://hdl.handle.net/10810/67870}
    \end{onecolentry}
\end{samepage}

\section{PROYECTOS}
\begin{onecolentry}
    \textbf{SAC Report Scraper}
\end{onecolentry}
\vspace{0.10 cm}
\begin{onecolentry}
    \begin{highlights}
        \item Desarrollé un bot que navegaba más de 30 informes y generaba alrededor de 5000 ejecuciones para generar los documentos de auditorías fiscales para la empresa.
        \item El ahorro fue equivalente a un equipo de 4 personas trabajando durante 3 semanas.
        \item Tecnologías: Python, Selenium, Git.
    \end{highlights}
\end{onecolentry}

\vspace{0.2 cm}
\begin{onecolentry}
    \textbf{ETL SAP S/4HANA Cloud - SAP R3 On Premise}
\end{onecolentry}
\vspace{0.10 cm}
\begin{onecolentry}
    \begin{highlights}
        \item El proyecto plantea una solución a un escenario donde el cliente posee una base de datos que se conecta directamente a la base de datos de SAP R3, que será sustituida por un SAP S/4HANA en la nube. La solución permite que la lógica de negocio contenida en la base de datos personalizada del cliente se mantenga mediante una carga incremental desde el nuevo S/4HANA al R3.
        \item Diseñé e implementé la arquitectura de la carga incremental de más de 80 tablas de SAP.
        \item Lideré un equipo de 2 técnicos en el desarrollo del programa ABAP de exportación de datos en formato XML.
        \item Tecnologías: SQL Server, PowerShell, Python, ABAP.
    \end{highlights}
\end{onecolentry}

\section*{IDIOMAS}
\begin{itemize}[noitemsep]
    \item Euskera: C1
    \item Inglés: C1
\end{itemize}
\end{document}
